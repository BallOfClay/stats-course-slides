\section{Common Distributions}

\begin{frame}
Some distributions are so common, they have been named and entered our shared
statistical conciousness.
\end{frame}
%

%
\begin{frame}
The two familiar random variables

\begin{itemize}
\item The number of heads seen in ten flips of a quarter.
\item The number of heads seen in ten flips of a dime.
\end{itemize}

Have a \textbf{binomial distribution}.

\end{frame}
%

%
\begin{frame}

\begin{align*}
P(\text{We get} & \text{ 2 heads in 10 flips of a quarter}) \\
%
&= {{10}\choose{2}} \times \left(\frac{1}{2} \right)^{10} \\
%
&= 0.044
\end{align*}

\end{frame}
%

%
\begin{frame}

The \textbf{binomial distribution} describes the number of events that happen in
a fixed number of \textbf{attempts} when the events \textbf{individually happen
with the same probability}.

Here the individual heads happen with probability $\frac{1}{2}$, and

\begin{align*}
P(\text{We get} & \text{ k heads in n flips of a quarter}) \\
%
&= {{n}\choose{k}} \times \left(\frac{1}{2} \right)^n
\end{align*}

\end{frame}
%

%
\begin{frame}
If the coin in \textbf{unfair}, so that the probability of an individual head is
$p$, then

\begin{align*}
P(\text{We get} & \text{ k heads in n flips of a quarter}) \\
%
&= {{n}\choose{k}} \times p^k \times (1 - p)^{n - k}
\end{align*}

\end{frame}
%

%
\begin{frame}

In the two examples

\begin{itemize}
\item The number of buses that arrive late to a stop in Seattle in a single
day.
\item The number of times my cat asks for food between 5 and 6 pm (when she is
always fed) in a given day.
\end{itemize}

We see a similarity: they are both about the number of times an event happens
\textbf{in a given span of time (or space)}.
\end{frame}
%

%
\begin{frame}
If we assume that the buses arrive at a fixed rate (but possibly unknown), and
the cat meows at a fixed rate, then these are both examples of the
\textbf{Poisson Distribution}.

$$ P(\text{Cat meows k times in one hour}) = e^{-\lambda} \frac{\lambda^k}{k!}
$$

The $\lambda$ above is the \textbf{rate the event occurs}.
\end{frame}
%

%
\begin{frame}
Suppose we observe the cat meow 5 times in ten minutes.  What is the probability
that the cat will not meow at all in the next five minutes?
\end{frame}
%

%
\begin{frame}
The rate the cat meows is:

$$ \lambda = \frac{5 \text{ meows}}{10 \text{ minuets}} = 5 \frac{ \text{
meows}}{\text{ 10 minuets}} $$

So using the Poisson equation

$$ P(\text{Cat meows zero times in ten minuets}) = e^{-5} \frac{5^0}{0!} = 0.007
$$
\end{frame}
%

%
\begin{frame}
What is the probability the cat meows zero times in the next hour?

$$ 
\lambda = \frac{5 \text{ meows}}{10 \text{ minuets}} = 30 \frac{ \text{
meows}}{\text{ 60 minuets}} = 30 \frac{ \text{
meows}}{\text{ hour}}
$$

$$ P(\text{Cat meows zero times in the next hour}) = e^{-30} \frac{5^0}{0!} =
9.3 \times 10^{-14}
$$

Its basically impossible.
\end{frame}
%

%
\begin{frame}
The $n$ in a binomial distribution and the $\lambda$ in Poisson distribution are
called \textbf{parameters}.

The general strategy for solving problems like is:

\begin{itemize}
\item Use the information in the statement of the problem to determine a likely
distribution for the quantity of interest.
\item Use the information in the problem to determine the values to use for the
parameters of this distribution.
\item Use the probability function of the distribution to compute the needed
probability.
\end{itemize}

\end{frame}
%

%
\begin{frame}

Other distributions to know:

\begin{itemize}
\item Normal Distribution.  \textbf{Parameters:} The mean $\mu$, and the
standard deviation $\sigma$.
\item Uniform Distribution. \textbf{Parameters:} The minimum $a$ and the maximum
$b$.
\item Exponential Distribution. \textbf{Parameters:} The rate $\alpha$.
\end{itemize}

\end{frame}
