\section{Independence}

\begin{frame}

Two events $A$ and $B$ are called \textbf{independent} when

$$ P(A \mid B) = P(A) $$

This means that \textbf{knowledge that B has or will occur does not change our
knowledge about whether A will occur}.

\end{frame}
%

%
\begin{frame}

Remember that the definition of conditional probability is

$$ P(A \mid B) = \frac{P(A \text{ and } B)}{P(B)} $$

Combining this with our definition of independence (so, below, $A$ and $B$ are
independent

$$ P(A) = P(A \mid B) = \frac{P(A \text{ and } B)}{P(B)} $$

Or, rearranging things

$$ P(A \text{ and } B) = P(A) P(B) $$

This equation is sometimes used as the \textbf{definition} of independence,
though I think it is less convincing.

\end{frame}
%

%
\begin{frame}

Common applications of independence generally go like this:

\begin{itemize}
\item We deduce from context that some event bears no influence on another.
\item We conclude that the two events are independent.
\item We use either of the two equations defining independence to fo
calculations.
\end{itemize}

\end{frame}
%

%
\begin{frame}

You roll a six sided die six times, what is the probability that you roll
\textbf{all
the possible numbers, in decreasing order}?

\end{frame}
%

%
\begin{frame}

Let's call the values of the rolls $R_1, R_2, R_3, R_4, R_5$ and $R_6$.  Then we
are looking for

$$P(R_1 = 6 \text{ and } R_2 = 5 \text{ and } R_3 = 4\text{ and } R_4 = 3 \text{
and } R_5 = 2 \text{ and } R_6 = 1)$$

Since each individual roll of the die does not influence the others, all size
rolls are independent.  This means we can break up and multiply

$$ P(R_1 = 6) \times P(R_2 = 5) \times P(R_3 = 4) \times P(R_4 = 3) \times P(R_5
= 2) \times P(R_6 = 1) $$

\end{frame}
%

%
\begin{frame}
Since each individual roll of the die has \textbf{six possible outcomes} and
\textbf{only one of them is the number we are looking for} we get

$$ P(\cdots) = \frac{1}{6} \times \frac{1}{6} \times \frac{1}{6} \times
\frac{1}{6} \times \frac{1}{6} \times \frac{1}{6} = \frac{1}{46656} $$
\end{frame}
%

%
\begin{frame}
Suppose you have a bucket with 5 red and 5 yellow balls in it, which you draw in
sequence, without replacement.  Are the events "You draw a red ball first" and 
"You draw a yellow ball second" independent?

\hfill

Suppose that it rains with probability $80 \%$ each day in seattle, and a winter
month has four work weeks.  What is the probability that there is a work week in
which it rains \textbf{every day}?  What assumptions do you have to make to
solve this problem?  Are they reasonable?
\end{frame}
