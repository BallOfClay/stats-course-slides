\documentclass{beamer}
\begin{document}
\title{Probability and Statistics Short Course: Day 1}   
\author{Matthew Drury} 
\date{\today} 

\frame{\titlepage} 

\frame{\frametitle{Table of contents}\tableofcontents} 

%%%%%%%%%%%%%%%%%%%%%%%%%%%%%%%%%%
\section{Introduction} 

%
\frame{\Large

What is Probability?

What is Statistics?

What is the difference?

}%

%
\frame{\Large

\textbf{Probability} refers to the study of patterns in a random process.

When we solve problems in probability we assume that all basic features of the random process are \textbf{known}, and our goal is to discover other, deeper features.

}%

%
\frame{\Large

For example:

If I have a coin which is \textbf{known} to land head exactly half of the time, it is a problem in \textbf{probability} to determine how often the coin will never land on heads over ten consecutive flips.

}%

%
\frame{\Large

\textbf{Statistics} refers to the study of random process where some basic features of the random process are \textbf{unknown}, and our goal is to \textbf{infer from observations} basic, hidden features of the random process.

}%

\frame{\Large

For example:

It is a problem in \textbf{statistics} to determine, when presented with a coin which has landed tails ten consecutive times, whether one should continue to believe it fair.

}%

%
\frame{\Large

In this course:

\textbf{Day 1 (Tuesday):} Basics of Probability.

\textbf{Day 2 (Thursday):} Basics of Statistics.

}%

%%%%%%%%%%%%%%%%%%%%%%%%%%%%%%%%%%

\section{Counting (Combinatorics)}

%
\frame{ 
The basic problem solving skill you need to solve problems in probability is \textbf{counting} (no, really).
}%

%
\frame{

For example:

\begin{itemize}
\item How many ways are there to arrange four letters of the alphabet?
\item How many ways are there to arrange four \textit{different} letters of the alphabet.
\item How many ways are there to arrange 25 math books on a bookshelf.
\item
\item
\end{itemize}

}%

%
\frame{

\textbf{Basic Counting Principle}

If a task can be accomplished as a series steps, then the number of outcomes of the task is the \textbf{product} of the number of outcomes of each individual step.

}%

%
\frame{

How many ways are there to arrange four letters of the alphabet?

Think: How can we accomplish this task...

{\center \_ \_ \_ \_}

}%

%
\frame{

How many ways are there to arrange four letters of the alphabet?

Think: How can we accomplish this task...

Pick the first letter, write it down.

$\Rightarrow$ Pick the second letter, write it down.

$\Rightarrow$ Pick the third letter, write it down.

$\Rightarrow$ Pick the fourth letter, write it down.

}%

%
\frame{

$$ 26 \times 26 \times 26 \times 26 = 456976 $$

}%

%
\frame{

How would this change if we could \textbf{not} re-use a letter?

}%

%
\frame{

The previous example is a common situation:  we are pulling from a pool of objects, and we \textbf{cannot} re-use an object once selected.

This is called \textbf{selection without replacement}.

}%

\frame{

The number of \textbf{ordered} selections \textbf{of k objects} without replacement \textbf{from a population for k objects} is called the \textbf{number of permutations of k objects taken from n}.

$$ P(n, k) = \underbrace{n \times (n-1) \times (n-2) \times \ldots \times (n - k + 1)}_{\text{k total factors}} $$

}%


%
\frame{

You have 25 math and stats books on a bookshelf.  How many ways are there to arrange these books in any order?

}%

%
\frame{

$$ 25 \times 24 \times 23 \times \ldots \times 2 \times 1 = 15511210043330985984000000 $$

}%

%
\frame{

What if we have a procedure in which the order of choices does not matter?

How many 5 card hands are possible when drawing from a standard 52 card deck?

Notice that the \textbf{order in which we draw cards is not important here}.

}%

%
\frame{

\textbf{Think:} To choose an \text{ordered} list of five cards I can \text{first} chose the five cards I want to use \text{and then} choose a way to order them.

$$ \text{\# of ordered hands} = \text{\# of unordered hands} \times \text{\# of ways to order five cards}$$

}%

%
\frame{

$$ 52 \times 51 \times 50 \times 49 \times 48 = \text{\# of unordered hands} \times 5 \times 4 \times 3 \times 2 \times 1 $$

}%

%
\frame{

$$ \text{\# of unordered hands} = \frac{52 \times 51 \times 50 \times 49 \times 48}{5 \times 4 \times 3 \times 2 \times 1} $$

}

%
\frame{

The number of \text{unordered} selections \textbf{of k objects} without replacement \textbf{from a population for k objects} is called the \textbf{number of combinations of k objects taken from n}.

$$ C(n, k) = \frac{P(n, k)}{P(k, k)} $$

}%

\frame{

A \textbf{full house} is a hand of five cards that has a \textbf{pair} of the same value, and a \textbf{three of a kind} of the same value.  How many hands of five cards are full houses?

}

%
\frame{

A \textbf{full house} is a hand of five cards that has a \textbf{pair} of the same value, and a \textbf{three of a kind} of the same value.  How many hands of five cards are full houses?

Choose a value for the pair

$\Rightarrow$ Choose two cards of that value

$\Rightarrow$ Chose a value for the three of a kind

$\Rightarrow$ Choose three cards of that value

}%

%
\frame{

A \textbf{full house} is a hand of five cards that has a \textbf{pair} of the same value, and a \textbf{three of a kind} of the same value.  How many hands of five cards are full houses?

Choose a value for the pair

$\Rightarrow$ Choose two cards of that value

$\Rightarrow$ Chose a value for the three of a kind

$\Rightarrow$ Choose three cards of that value

$$ 13 \times C(4, 2) \times 12 \times C(4, 3) = 13 \times 6 \times 12 \times 4 = 3744 $$

}%

\end{document}